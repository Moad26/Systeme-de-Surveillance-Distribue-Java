\documentclass[aspectratio=169, 12pt]{beamer}

% ============================================================================
% THÈME ET CONFIGURATION
% ============================================================================
\usetheme{Madrid}
\usecolortheme{whale}
\usepackage[utf8]{inputenc}
\usepackage[french]{babel}
\usepackage[T1]{fontenc}
\usepackage{graphicx}
\usepackage{tikz}
\usepackage{listings}
\usepackage{xcolor}

% Personnalisation des couleurs
\definecolor{primary}{RGB}{0, 82, 147}
\definecolor{secondary}{RGB}{0, 128, 128}
\definecolor{accent}{RGB}{255, 127, 0}
\definecolor{success}{RGB}{39, 174, 96}
\definecolor{danger}{RGB}{231, 76, 60}

\setbeamercolor{title}{fg=white, bg=primary}
\setbeamercolor{frametitle}{fg=white, bg=primary}
\setbeamercolor{block title}{bg=secondary, fg=white}
\setbeamercolor{block body}{bg=secondary!10}

% Configuration du code
\lstset{
    basicstyle=\ttfamily\scriptsize,
    keywordstyle=\color{primary}\bfseries,
    commentstyle=\color{gray},
    stringstyle=\color{success},
    breaklines=true,
    frame=single,
    rulecolor=\color{gray!30},
    language=Java,
    showstringspaces=false
}

% Retirer les éléments de navigation
\setbeamertemplate{navigation symbols}{}
\setbeamertemplate{footline}[frame number]

% ============================================================================
% INFORMATIONS DU DOCUMENT
% ============================================================================
\title{\textbf{Système de Surveillance Distribué}}
\subtitle{Monitoring Temps Réel Multi-Agent}
\author{
    \textbf{Membre A} \and \textbf{Membre B} \and \textbf{Membre C}
}
\institute{
    Université [Nom] \\
    \smallskip
    \textit{Projet de Systèmes Distribués}
}
\date{Année Universitaire 2025-2026}

\begin{document}

% ============================================================================
% SLIDE 1 : PAGE DE TITRE
% ============================================================================
\begin{frame}[plain]
    \maketitle
\end{frame}

% ============================================================================
% SLIDE 2 : SOMMAIRE
% ============================================================================
\begin{frame}{Sommaire}
    \tableofcontents
\end{frame}

% ============================================================================
% SECTION 1 : INTRODUCTION
% ============================================================================
\section{Introduction \& Problématique}

% ============================================================================
% SLIDE 3 : PROBLÉMATIQUE
% ============================================================================
\begin{frame}{Problématique}
    \begin{columns}[T]
        \begin{column}{0.55\textwidth}
            \textbf{Défis du Monitoring IT}
            \begin{itemize}
                \item \textcolor{danger}{\textbf{Pannes coûteuses}} : Indisponibilité système
                \item \textcolor{danger}{\textbf{Détection tardive}} : Alertes après incident
                \item \textcolor{danger}{\textbf{Silos d'information}} : Données éparpillées
                \item \textcolor{danger}{\textbf{Évolutivité}} : Centaines de machines
            \end{itemize}
            
            \vspace{0.5cm}
            \textbf{Notre Objectif}
            \begin{itemize}
                \item Supervision \textbf{temps réel} centralisée
                \item Alertes \textbf{proactives} configurables
                \item Architecture \textbf{distribuée} et \textbf{scalable}
            \end{itemize}
        \end{column}
        
        \begin{column}{0.4\textwidth}
            \begin{block}{Métriques Surveillées}
                \begin{itemize}
                    \item CPU Usage (\%)
                    \item RAM Usage (\%)
                    \item Disk Usage (\%)
                \end{itemize}
            \end{block}
            
            \begin{alertblock}{Niveaux d'Alerte}
                \begin{itemize}
                    \item \textcolor{accent}{WARNING} (seuil 1)
                    \item \textcolor{danger}{CRITICAL} (seuil 2)
                \end{itemize}
            \end{alertblock}
        \end{column}
    \end{columns}
\end{frame}

% ============================================================================
% SECTION 2 : ARCHITECTURE
% ============================================================================
\section{Architecture Globale}

% ============================================================================
% SLIDE 4 : ARCHITECTURE GLOBALE
% ============================================================================
\begin{frame}{Architecture Globale}
    \begin{center}
    \begin{tikzpicture}[
        box/.style={rectangle, draw=primary, fill=primary!10, rounded corners, minimum width=2.2cm, minimum height=1cm, align=center, font=\footnotesize},
        serverbox/.style={rectangle, draw=secondary, fill=secondary!15, rounded corners, minimum width=3cm, minimum height=2cm, align=center},
        arrow/.style={->, thick, >=stealth, color=gray!70},
        label/.style={font=\tiny, color=gray}
    ]
        % Agents (gauche)
        \node[box] (agent1) at (-5, 1.5) {Agent 1\\CPU/RAM/Disk};
        \node[box] (agent2) at (-5, 0) {Agent 2\\CPU/RAM/Disk};
        \node[box] (agent3) at (-5, -1.5) {Agent N\\CPU/RAM/Disk};
        
        % Serveur (centre)
        \node[serverbox] (server) at (0, 0) {
            \textbf{Serveur Central}\\[2pt]
            \scriptsize RMI Registry\\
            \scriptsize UDP/TCP Listeners\\
            \scriptsize DataManager
        };
        
        % UI Clients (droite)
        \node[box] (ui1) at (5, 0.8) {Client UI\\Dashboard};
        \node[box] (ui2) at (5, -0.8) {Client UI\\Admin};
        
        % Flèches UDP (métriques)
        \draw[arrow, color=secondary] (agent1) -- node[above, label] {UDP} (-1.5, 0.8);
        \draw[arrow, color=secondary] (agent2) -- node[above, label] {UDP} (-1.5, 0);
        \draw[arrow, color=secondary] (agent3) -- node[below, label] {UDP} (-1.5, -0.8);
        
        % Flèches TCP (alertes)
        \draw[arrow, color=danger, dashed] (agent1.east) to[bend left=15] node[above, label, yshift=3pt] {TCP Alert} (-1.5, 0.3);
        
        % Flèches RMI
        \draw[arrow, color=primary] (1.5, 0.5) -- node[above, label] {RMI} (ui1);
        \draw[arrow, color=primary] (1.5, -0.5) -- node[below, label] {RMI} (ui2);
        
        % Légende
        \node[label] at (0, -2.5) {\textcolor{secondary}{\rule{0.5cm}{2pt}} UDP Métriques \quad \textcolor{danger}{- - -} TCP Alertes \quad \textcolor{primary}{\rule{0.5cm}{2pt}} RMI Client};
    \end{tikzpicture}
    \end{center}
    
    \vspace{-0.3cm}
    \begin{columns}
        \begin{column}{0.33\textwidth}
            \centering\footnotesize\textbf{4 Modules Maven}\\monitoring-common/agent/server/ui
        \end{column}
        \begin{column}{0.33\textwidth}
            \centering\footnotesize\textbf{Multi-protocole}\\UDP + TCP + RMI
        \end{column}
        \begin{column}{0.33\textwidth}
            \centering\footnotesize\textbf{Temps Réel}\\Collecte chaque 5s
        \end{column}
    \end{columns}
\end{frame}

% ============================================================================
% SECTION 3 : FOCUS TECHNIQUE
% ============================================================================
\section{Focus Technique}

% ============================================================================
% SLIDE 5 : FOCUS AGENT - PATTERN STRATEGY
% ============================================================================
\begin{frame}[fragile]{Focus Agent : Pattern Strategy \& Threads}
    \begin{columns}[T]
        \begin{column}{0.45\textwidth}
            \textbf{Pattern Strategy}
            
            \begin{block}{Interface ICollector}
\begin{lstlisting}
public interface ICollector {
    double collect();
    String getName();
}
\end{lstlisting}
            \end{block}
            
            \textbf{Implémentations :}
            \begin{itemize}
                \item \texttt{CpuCollector}
                \item \texttt{MemoryCollector}
                \item \texttt{DiskCollector}
            \end{itemize}
            
            \textcolor{success}{\textbf{Extensible}} : Ajout facile de collecteurs
        \end{column}
        
        \begin{column}{0.5\textwidth}
            \textbf{Planification Périodique}
            
            \begin{block}{ScheduledExecutorService}
\begin{lstlisting}
scheduler.scheduleAtFixedRate(
    this::collectAndSend,
    0,    // initial delay
    5,    // period
    TimeUnit.SECONDS
);
\end{lstlisting}
            \end{block}
            
            \textbf{Avantages :}
            \begin{itemize}
                \item Thread-safe et non-bloquant
                \item Pool de threads géré
                \item Arrêt propre avec \texttt{shutdown()}
            \end{itemize}
        \end{column}
    \end{columns}
\end{frame}

% ============================================================================
% SLIDE 6 : FOCUS COMMUNICATION - UDP vs TCP
% ============================================================================
\begin{frame}{Focus Communication : Choix des Protocoles}
    \begin{columns}[T]
        \begin{column}{0.48\textwidth}
            \begin{block}{UDP pour les Métriques}
                \textbf{Pourquoi UDP ?}
                \begin{itemize}
                    \item \textcolor{success}{Latence minimale} (1ms)
                    \item \textcolor{success}{Pas de handshake}
                    \item \textcolor{success}{Scalable} : 10000+ paquets/s
                    \item Perte acceptable : métrique suivante dans 5s
                \end{itemize}
                
                \vspace{0.2cm}
                \textbf{Implémentation :}
                \begin{itemize}
                    \item \texttt{UdpSender} (Agent)
                    \item \texttt{UdpListener} (Serveur)
                    \item Port 9876
                \end{itemize}
            \end{block}
        \end{column}
        
        \begin{column}{0.48\textwidth}
            \begin{alertblock}{TCP pour les Alertes}
                \textbf{Pourquoi TCP ?}
                \begin{itemize}
                    \item \textcolor{danger}{Fiabilité garantie} (ACK)
                    \item \textcolor{danger}{Ordre préservé}
                    \item \textcolor{danger}{Intégrité} (checksum)
                    \item Alerte CRITICAL jamais perdue
                \end{itemize}
                
                \vspace{0.2cm}
                \textbf{Implémentation :}
                \begin{itemize}
                    \item \texttt{TcpClient} (Agent)
                    \item \texttt{TcpAlertHandler} + ThreadPool
                    \item Port 9877
                \end{itemize}
            \end{alertblock}
        \end{column}
    \end{columns}
\end{frame}

% ============================================================================
% SLIDE 7 : FOCUS RMI
% ============================================================================
\begin{frame}[fragile]{Focus Serveur : Service RMI}
    \begin{columns}[T]
        \begin{column}{0.48\textwidth}
            \textbf{Java RMI}
            
            \begin{block}{Interface Distante}
\begin{lstlisting}
public interface IMonitoringService 
        extends Remote {
    List<String> getActiveAgents();
    List<Metric> getMetrics(...);
    List<Alert> getAlerts(...);
    MetricStatistics getStatistics(...);
    String authenticate(user, pass);
}
\end{lstlisting}
            \end{block}
            
            \textbf{Pourquoi RMI ?}
            \begin{itemize}
                \item Appel de méthode transparent
                \item Sérialisation automatique
                \item Intégré à Java
            \end{itemize}
        \end{column}
        
        \begin{column}{0.48\textwidth}
            \textbf{Structures Thread-Safe}
            
            \begin{itemize}
                \item \texttt{ConcurrentHashMap} pour les métriques
                \item \texttt{CopyOnWriteArrayList} pour les alertes
                \item \texttt{ExecutorService} pour TCP
            \end{itemize}
            
            \vspace{0.3cm}
            \begin{exampleblock}{Persistance JSON}
                \begin{itemize}
                    \item \texttt{metrics/[agentId].json}
                    \item \texttt{alerts.json}
                    \item \texttt{users.json} (SHA-256)
                    \item \texttt{alert\_configs.json}
                \end{itemize}
            \end{exampleblock}
        \end{column}
    \end{columns}
\end{frame}

% ============================================================================
% SECTION 4 : INTERFACE UTILISATEUR
% ============================================================================
\section{Interface Utilisateur}

% ============================================================================
% SLIDE 8 : FOCUS UI - JAVAFX
% ============================================================================
\begin{frame}[fragile]{Interface JavaFX : Dashboard Temps Réel}
    \begin{columns}[T]
        \begin{column}{0.45\textwidth}
            \textbf{Architecture MVC}
            
            \begin{itemize}
                \item \textbf{Model} : \texttt{Metric}, \texttt{Alert}, \texttt{User}
                \item \textbf{View} : FXML (déclaratif)
                \item \textbf{Controller} : \texttt{DashboardController}
            \end{itemize}
            
            \vspace{0.3cm}
            \textbf{Rafraîchissement Automatique}
            
            \begin{block}{Timeline JavaFX}
\begin{lstlisting}
Timeline timeline = new Timeline(
  new KeyFrame(Duration.seconds(2),
    e -> refreshData()));
timeline.setCycleCount(INDEFINITE);
timeline.play();
\end{lstlisting}
            \end{block}
        \end{column}
        
        \begin{column}{0.5\textwidth}
            \textbf{Fonctionnalités UI}
            
            \begin{itemize}
                \item Graphiques temps réel (CPU/RAM/Disk)
                \item Tableau des alertes avec filtres
                \item Statistiques : moy, min, max, écart-type
                \item Authentification (ADMIN/OPERATOR/VIEWER)
                \item Configuration des seuils d'alerte
                \item Export CSV / JSON
                \item Recherche et filtrage avancé
            \end{itemize}
            
            \vspace{0.2cm}
            \begin{block}{\texttt{Platform.runLater()}}
                Mises à jour sur le thread JavaFX
            \end{block}
        \end{column}
    \end{columns}
\end{frame}

% ============================================================================
% SECTION 5 : DÉMONSTRATION
% ============================================================================
\section{Démonstration}

% ============================================================================
% SLIDE 9 : DÉMONSTRATION
% ============================================================================
\begin{frame}[c]{Démonstration Live}
    \begin{center}
        {\Huge $\triangleright$} \\[0.5cm]
        \Large \textbf{Démonstration du Système}
        
        \vspace{1cm}
        
        \begin{columns}
            \begin{column}{0.3\textwidth}
                \centering
                {\Large [S]} \\
                \normalsize \textbf{1. Serveur}\\
                \footnotesize Démarrage RMI\\+ Listeners
            \end{column}
            \begin{column}{0.3\textwidth}
                \centering
                {\Large [A]} \\
                \normalsize \textbf{2. Agent(s)}\\
                \footnotesize Collecte \& Envoi\\métriques
            \end{column}
            \begin{column}{0.3\textwidth}
                \centering
                {\Large [UI]} \\
                \normalsize \textbf{3. Client UI}\\
                \footnotesize Dashboard\\temps réel
            \end{column}
        \end{columns}
        
        \vspace{0.8cm}
        \textit{\small Observation : alertes automatiques lors de dépassement de seuil}
    \end{center}
\end{frame}

% ============================================================================
% SECTION 6 : CONCLUSION
% ============================================================================
\section{Conclusion \& Perspectives}

% ============================================================================
% SLIDE 10 : BILAN TECHNIQUE
% ============================================================================
\begin{frame}{Bilan Technique}
    \begin{columns}[T]
        \begin{column}{0.48\textwidth}
            \begin{block}{Objectifs Atteints}
                \begin{itemize}
                    \item[\textcolor{success}{$\checkmark$}] Monitoring temps réel multi-agents
                    \item[\textcolor{success}{$\checkmark$}] Architecture distribuée (UDP/TCP/RMI)
                    \item[\textcolor{success}{$\checkmark$}] Pattern Strategy extensible
                    \item[\textcolor{success}{$\checkmark$}] Alertes configurables
                    \item[\textcolor{success}{$\checkmark$}] Dashboard JavaFX moderne
                    \item[\textcolor{success}{$\checkmark$}] Authentification et rôles
                    \item[\textcolor{success}{$\checkmark$}] Statistiques avancées
                    \item[\textcolor{success}{$\checkmark$}] Export CSV/JSON
                \end{itemize}
            \end{block}
        \end{column}
        
        \begin{column}{0.48\textwidth}
            \begin{alertblock}{Compétences Acquises}
                \begin{itemize}
                    \item Programmation réseau (Sockets)
                    \item Java RMI et invocation distante
                    \item Concurrence Java
                    \item Patterns de conception (Strategy, MVC)
                    \item JavaFX et interfaces modernes
                    \item Architecture modulaire Maven
                \end{itemize}
            \end{alertblock}
        \end{column}
    \end{columns}
\end{frame}

% ============================================================================
% SLIDE 11 : PERSPECTIVES
% ============================================================================
\begin{frame}{Perspectives d'Amélioration}
    \begin{columns}[T]
        \begin{column}{0.48\textwidth}
            \textbf{Évolutions Futures}
            
            \begin{itemize}
                \item Base de données SQL pour scalabilité
                \item API REST (Spring Boot)
                \item Conteneurisation Docker
                \item Déploiement cloud (AWS/Azure)
                \item Machine Learning pour prédiction
            \end{itemize}
        \end{column}
        
        \begin{column}{0.48\textwidth}
            \textbf{Fonctionnalités Additionnelles}
            
            \begin{itemize}
                \item Application mobile
                \item Notifications email/SMS
                \item Métriques réseau
                \item Rapports automatiques PDF
                \item Plugins Grafana/Prometheus
            \end{itemize}
        \end{column}
    \end{columns}
\end{frame}

% ============================================================================
% SLIDE 12 : QUESTIONS
% ============================================================================
\begin{frame}[c]{Questions \& Discussion}
    \begin{center}
        {\Huge ?} \\[0.8cm]
        \Large \textbf{Merci de votre attention !}
        
        \vspace{1cm}
        
        \large \textit{Questions ?}
        
        \vspace{1.5cm}
        
        \footnotesize
        Code source disponible sur demande \\
        Documentation complète dans le rapport
    \end{center}
\end{frame}

\end{document}
